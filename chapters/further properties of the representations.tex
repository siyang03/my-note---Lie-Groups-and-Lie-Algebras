\chapter{further properties of the representations}
\section{the structure of weights}
\begin{itemize}
	\item \textbf{theorem:} 对于 semisimple Lie algebra $\mathfrak{g}$ 的一个 irreducible finite-dim. rep. $(\pi_\mu, V^\mu)$, 其 highest weight 为 $\mu$, 那么, integral element $\lambda$ 是其 weight $\iff \lambda$ 满足以下两个条件,
	\begin{enumerate}
		\item $\lambda \in \mathrm{Conv}(W \ket{\mu})$,
		
		\item $\mu - \lambda$ 可以表示成 roots 的整数线性组合.
	\end{enumerate}
	
	\begin{tcolorbox}[title=proof:]
		\begin{itemize}
			\item \textbf{"no holes" lemma:} 对于一个 semisimple Lie algebra $\mathfrak{g}$ 的 finite-dim. rep. $(\pi, V)$, $\lambda$ 是它的一个 weight, 那么, 对于一个 root $\alpha$ 满足 $\braket{\lambda, \alpha} > 0$, 有,
			\begin{equation}
				\lambda - i \alpha, i \in \{0, 1, \cdots, \braket{\lambda, H_\alpha}\}
			\end{equation}
			都是 weights, (也就是 $\lambda, \lambda - \alpha, \cdots, s_\alpha \ket{\lambda}$).
			
			\noindent\hdashrule[0.5ex]{\linewidth}{0.5pt}{1mm} % horizontal dashed line
			
			\textbf{proof:}
			
			考虑如下 weight spaces 的直和,
			\begin{equation}
				V \supset U = \bigoplus_{i \in \mathbb{Z}} V_{\lambda - i \alpha}
			\end{equation}
			(线性独立证明见 \eqref{A.3.9}), 那么 $U$ 在 $\mathfrak{s}^\alpha = \mathrm{span}(H_\alpha, A_\alpha, B_\alpha)$ 的作用下保持不变. 并且注意到 $V_{\lambda - i \alpha}$ 是以,
			\begin{equation}
				\braket{\lambda, H_\alpha} - 2 i
			\end{equation}
			为本征值的 $H_\alpha$ 的本征空间, 根据 \eqref{10.1.6} (不需要 irreducibility) 可知 $\braket{\lambda, H_\alpha}, \cdots, - \braket{\lambda, H_\alpha}$ 都是本征值.
		\end{itemize}
		
		\noindent\rule[0.5ex]{\linewidth}{0.5pt} % horizontal line
		
		首先考虑 $\lambda$ 是 dominant integral (结合条件 1 implies $\lambda \preceq \mu$), 来证明 $\Longleftarrow$ (去除条件 finite-dim.).
	\end{tcolorbox}
\end{itemize}

\section{the Casimir element}
\begin{itemize}
	\item \textbf{\textbf{def.:}} the 2nd-order Casimir operator is,
	\begin{equation}
		C_2 = - B^{i j} A_i \otimes A_j
	\end{equation}
	where $B^{i j} = B^{- 1}_{i j}$.
	\begin{itemize}
		\item the 2nd-order Casimir operator commutes with all the generators.
		
		\begin{tcolorbox}[title=proof:]
			\begin{align}
				[C_2, A_k] &= - B^{i j} [A_i A_j, A_k] \notag \\
				&= - B^{i j} (- {f_{j k}}^l A_i A_l - {f_{i k}}^l A_l A_j)
			\end{align}
			notice that $B^{i j}$ is symmetric, so,
			\begin{align}
				[C_2, A_k] &= - B^{i j} (- {f_{i k}}^l A_j A_l - {f_{i k}}^l A_l A_j) \notag \\
				&= B^{i j} {f_{i k}}^l (A_j A_l + A_l A_j) \notag \\
				&= \underbrace{B^{i j} B^{l m} (A_j A_l + A_l A_j)}_{\text{symmetric about } (i, m)} f_{i k m} = 0
			\end{align}
		\end{tcolorbox}
	\end{itemize}
\end{itemize}
