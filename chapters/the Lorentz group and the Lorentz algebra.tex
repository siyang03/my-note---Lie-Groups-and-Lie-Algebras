\chapter{the Lorentz group and the Lorentz algebra}
\begin{itemize}
	\item Wikipedia: \href{https://en.wikipedia.org/wiki/Representation_theory_of_the_Lorentz_group}{Representation theory of the Lorentz group}.
\end{itemize}

\section{indefinite orthogonal groups}
\begin{itemize}
	\item $\mathrm{O}(p, q) = \{\Lambda \in \mathcal{M}_n(\mathbb{R}) | \Lambda^T \eta \Lambda = \eta\}$ is called the \href{https://en.wikipedia.org/wiki/Indefinite_orthogonal_group}{indefinite orthogonal group}, where $n = p + q$ and,
	\begin{equation}
		\eta = \mathrm{diag}(\underbrace{+ 1, \cdots, + 1}_{p}, \underbrace{- 1, \cdots, - 1}_{q})
	\end{equation}
	\begin{itemize}
		\item 将 $\Lambda$ 矩阵视作一组列向量 $(\lambda_1, \cdots, \lambda_n)$, 那么,
		\begin{equation}
			\eta(\lambda_\mu, \lambda_\nu) = \eta_{\mu \nu}
		\end{equation}
		即 $n$ 个互相正交的向量.
		
		\item $\dim \mathrm{O}(p, q) = \frac{n (n - 1)}{2}$.
		
		\item 可以证明, 对于,
		\begin{equation}
			\Lambda = \begin{pmatrix}
				A & B \\
				C & D
			\end{pmatrix}
		\end{equation}
		有 $\det \Lambda = \frac{\det A}{\det D}$, 且 $|\det A|, |\det D| \geq 1$.
		
		\begin{tcolorbox}[title=proof:]
			分块矩阵满足,
			\begin{equation} \label{11.3.4}
				\begin{dcases}
					A^T B = C^T D \\
					A^T A - C^T C = I_{p \times p} \\
					D^T D - B^T B = I_{q \times q}
				\end{dcases}
			\end{equation}
			如果 $\det A \neq 0$, 那么,
			\begin{equation}
				\det \Lambda = \det(A) \det(D - C A^{- 1} B)
			\end{equation}
			对 \eqref{11.3.4} 的第一行做变换, 得到,
			\begin{equation}
				A^{- 1} = C^{- 1} (D^T)^{- 1} B^T \Longrightarrow C A^{- 1} B = (D^T)^{- 1} B^T B
			\end{equation}
			再代入 \eqref{11.3.4} 的第三行, 得到 $C A^{- 1} B = D - (D^T)^{- 1}$, 所以...
			
			\noindent\rule[0.5ex]{\linewidth}{0.5pt} % horizontal line
			
			由 \eqref{11.3.4} 的第二行,
			\begin{equation}
				{\det}^2 A = \det(I + C^T C) \overset{\textcolor{red}{(?)}}{\geq} 1
			\end{equation}
		\end{tcolorbox}
	\end{itemize}
	
	\item $\mathrm{O}(p, q)$ 具有如下子群,
	\begin{equation}
		\begin{dcases}
			\mathrm{SO}(p, q) = \{\Lambda \in \mathrm{O}(p, q) | \det \Lambda = 1\} \\
			\mathrm{SO}_+(p, q) = \{\Lambda \in \mathrm{SO}(p, q) | \det A \geq 1\} \\
			\mathrm{O}_+(p, q) = \{\Lambda \in \mathrm{O}(p, q) | \det A \geq 1\} \\
			\mathrm{O}_-(p, q) = \{\Lambda \in \mathrm{O}(p, q) | \det D \geq 1\}
		\end{dcases}
	\end{equation}
	且有如下四个连通分支,
	\begin{equation}
		\mathrm{SO}_\pm(p, q) \quad \text{and} \quad \mathrm{O'}_\pm(p, q) = \{\det \Lambda = - 1, \det A \geq 1 \ \text{or} \ \det A \leq - 1\}
	\end{equation}
\end{itemize}

\subsection{universal cover and \texorpdfstring{$\mathrm{Spin}(p, q)$}{Spin(p, q)}}
\begin{itemize}
	\item 注意到 $\mathrm{SO}_+(p, q)$ 是连通, 但不是单连通, 回顾 subsection \ref{subsection 5.1.1}, $\mathrm{SO}_+(p, q)$ 的表示不能唯一地由 $\mathfrak{so}(p, q)$ 的表示来确定.
	
	\item 回顾 subsection \ref{subsection 5.1.2}, $\mathrm{Spin}(p, q)$ 的李代数与 $\mathfrak{so}(p, q)$ 同构, 且是 simply connected, 是 $\mathrm{SO}_+(p, q)$ 的 universal cover.
\end{itemize}

\section{the Lorentz group}
\begin{itemize}
	\item $\mathrm{L} = \mathrm{O}(3, 1)$ is called the Lorentz group.
	
	\item 有 3 个 rotations,
	\begin{align}
		& R(\omega_{x y}) = \begin{pmatrix}
			1 & & & \\
			& \cos \omega_{x y} & - \sin \omega_{x y} & \\
			& \sin \omega_{x y} & \cos \omega_{x y} & \\
			& & & 1
		\end{pmatrix} \quad R(\omega_{y z}) = \begin{pmatrix}
			1 & & & \\
			& 1 & & \\
			& & \cos \omega_{y z} & - \sin \omega_{y z} \\
			& & \sin \omega_{y z} & \cos \omega_{y z}
		\end{pmatrix} \notag \\
		& R(\omega_{z x}) = \begin{pmatrix}
			1 & & & \\
			& \cos \omega_{z x} & & \sin \omega_{z x} \\
			& & 1 & \\
			& - \sin \omega_{z x} & & \cos \omega_{z x}
		\end{pmatrix}
	\end{align}
	和 3 个 boosts,
	\begin{align}
		& B(\omega_{t x}) = \begin{pmatrix}
			\cosh \omega_{t x} & \sinh \omega_{t x} & & \\
			\sinh \omega_{t x} & \cosh \omega_{t x} & & \\
			& & 1 & \\
			& & & 1
		\end{pmatrix} \quad B(\omega_{t y}) = \begin{pmatrix}
			\cosh \omega_{t y} & & \sinh \omega_{t y} & \\
			& 1 & & \\
			\sinh \omega_{t y} & & \cosh \omega_{t y} & \\
			& & & 1
		\end{pmatrix} \notag \\
		& B(\omega_{t z}) = \begin{pmatrix}
			\cosh \omega_{t z} & & & \sinh \omega_{t z} \\
			& 1 & & \\
			& & 1 & \\
			\sinh \omega_{t z} & & & \cosh \omega_{t z}
		\end{pmatrix}
	\end{align}
\end{itemize}

\subsection{parity and time reversal}
\begin{itemize}
	\item $\mathrm{O}(3, 1)$ 有 4 个连通分支,
	\begin{equation}
		I \in \mathrm{SO}_+(3, 1) \quad P T \in \mathrm{SO}_-(3, 1) \quad P \in \mathrm{O}'_+(3, 1) \quad T \in \mathrm{O}'_-(3, 1)
	\end{equation}
	其中,
	\begin{equation}
		P = \mathrm{diag}(+ 1, - 1, - 1, - 1) \quad T = \mathrm{diag}(- 1, + 1, + 1, + 1)
	\end{equation}
	另外, $\eta P \eta = P, \eta T \eta = T$.
\end{itemize}

\section{the Lorentz algebra}
\begin{itemize}
	\item $\mathfrak{so}(3, 1) = \{A \in \mathcal{M}_4(\mathbb{R}) | A^T = - \eta A \eta\}$.
	
	\item 选择如下 6 个与 rotations and boosts 对应的基矢,
	\begin{align}
		& J^{1 2} = \frac{d}{d\omega_{x y}} R(\omega_{x y}) = \begin{pmatrix}
			0 & & & \\
			& 0 & - 1 & \\
			& 1 & 0 & \\
			& & & 0
		\end{pmatrix} \quad J^{2 3} = \begin{pmatrix}
			0 & & & \\
			& 0 & & \\
			& & 0 & - 1 \\
			& & 1 & 0
		\end{pmatrix} \quad J^{3 1} = \begin{pmatrix}
			0 & & & \\
			& 0 & & 1 \\
			& & 0 & \\
			& - 1 & & 0
		\end{pmatrix} \notag \\
		& J^{0 1} = \begin{pmatrix}
			0 & 1 & & \\
			1 & 0 & & \\
			& & 0 & \\
			& & & 0
		\end{pmatrix} \quad J^{0 2} = \begin{pmatrix}
			0 & & 1 & \\
			& 0 & & \\
			1 & & 0 & \\
			& & & 0
		\end{pmatrix} \quad J^{0 3} = \begin{pmatrix}
			0 & & & 1 \\
			& 0 & & \\
			& & 0 & \\
			1 & & & 0
		\end{pmatrix}
	\end{align}
	在某个方向做 rotation 或 boost 可以一般地写为 $\Lambda = e^{\frac{1}{2} \omega_{\mu \nu} J^{\mu \nu}}$ (另外, $\mathrm{SO}(3, 1)$ is compact, 它的连通分支内, 所有元素都可以写成指数形式, 原因见 \eqref{4.2.8}).
	\begin{itemize}
		\item $J^{\mu \nu}$ 可以一般地写作如下形式,
		\begin{equation}
			\tensor{(J^{\mu \nu})}{^\rho_\sigma} = - 2 \eta^{[\mu| \rho} \tensor{\delta}{^{|\nu]}_\sigma}
		\end{equation}
		
		\item 存在如下对易关系,
		\begin{equation}
			[J^{\mu \nu}, J^{\rho \sigma}] = \eta^{\mu \rho} J^{\nu \sigma} + \eta^{\nu \sigma} J^{\mu \rho} - \eta^{\mu \sigma} J^{\nu \rho} - \eta^{\nu \rho} J^{\mu \sigma}
		\end{equation}
		
		\item 注意, $\mathrm{SO}_+(3, 1)$ 不是单连通, 所以 $\mathrm{SO}_+(3, 1)$ 的表示 $\Pi$ 不能由 $\pi$ 唯一确定, 例如可能出现如下情况,
		\begin{align}
			& \Lambda^1 = e^{\frac{1}{2} \omega^1_{\mu \nu} J^{\mu \nu}} \quad \Lambda^2 = e^{\frac{1}{2} \omega^2_{\mu \nu} J^{\mu \nu}} \Longrightarrow \Lambda^3 = \Lambda^1 \Lambda^2 = e^{\frac{1}{2} \omega^3_{\mu \nu} J^{\mu \nu}} \notag \\
			& \multicolumn{1}{c}{$\Updownarrow$ but} \notag \\
			& e^{\frac{1}{2} \omega^1_{\mu \nu} \pi(J^{\mu \nu})} e^{\frac{1}{2} \omega^2_{\mu \nu} \pi(J^{\mu \nu})} \neq e^{\frac{1}{2} \omega^3_{\mu \nu} \pi(J^{\mu \nu})}
		\end{align}
	\end{itemize}
\end{itemize}

\section{roots and root spaces of the Lorentz algebra}
\begin{itemize}
	\item 考虑 $\mathfrak{so}(4, \mathbb{C})$ 的 Dynkin diagram, $D_2$, (见 section \ref{6.7}), 可见 $\mathfrak{so}(4, \mathbb{C}) \simeq \mathfrak{sl}(2, \mathbb{C}) \oplus \mathfrak{sl}(2, \mathbb{C})$.
	\begin{itemize}
		\item 因此, $\mathfrak{so}(3, 1)$ 的 irreducible rep. 是 $\text{spin-} j_1 \oplus \text{spin-} j_2$, 用 $(j_1, j_2)$ 表示.
	\end{itemize}
	
	\noindent\rule[0.5ex]{\linewidth}{0.5pt} % horizontal line
	
	\item 参考 subsection \ref{subsection 6.7.2}, $\mathfrak{so}(3, 1)$ 的 maximal commutative subalgebra 为,
	\begin{equation}
		\mathfrak{t} = \mathrm{span}(J^{0 2}, J^{3 1}) \Longrightarrow \mathfrak{h} = \mathfrak{t}_\mathbb{C}
	\end{equation}
	仿照 \eqref{6.7.5}, 定义内积 $\braket{A, B} = \frac{1}{2} \mathrm{tr}(A^\dag B)$, 有,
	\begin{equation}
		\braket{\left( \begin{array}{cc|cc}
				& & - a & \\
				& & & b \\
				\hline
				- a & & & \\
				& - b & &
			\end{array} \right), \left( \begin{array}{cc|cc}
				& & - c & \\
				& & & d \\
				\hline
				- c & & & \\
				& - d & &
			\end{array} \right)
		} = a^* c + b^* d
	\end{equation}
	
	\item 令 $\alpha = - J^{0 2} + i J^{3 1}, \beta = - J^{0 2} - i J^{3 1}$, 那么有 coroots $H_\alpha = \alpha, H_\beta = \beta$, 且,
	\begin{equation}
		\begin{dcases}
			A_\alpha = - \frac{i}{2} ((J^{1 2} + i J^{2 3}) - (J^{0 1} - i J^{0 3})) \quad B_\alpha = - \frac{i}{2} ((J^{1 2} - i J^{2 3}) + (J^{0 1} + i J^{0 3})) \\
			A_\beta = - \frac{i}{2} ((J^{1 2} - i J^{2 3}) - (J^{0 1} + i J^{0 3})) \quad B_\beta = - \frac{i}{2} ((J^{1 2} + i J^{2 3}) + (J^{0 1} - i J^{0 3}))
		\end{dcases}
	\end{equation}
	或者,
	\begin{equation}
		\begin{dcases}
			J^{1 2} = - \frac{A_\alpha + B_\alpha + A_\beta + B_\beta}{2 i} \\
			J^{2 3} = \frac{A_\alpha - B_\alpha - A_\beta + B_\beta}{2}
		\end{dcases} \quad \begin{dcases}
			J^{0 1} = \frac{A_\alpha - B_\alpha + A_\beta - B_\beta}{2 i} \\
			J^{0 3} = \frac{A_\alpha + B_\alpha - A_\beta - B_\beta}{2}
		\end{dcases}
	\end{equation}
	\begin{itemize}
		\item 为了让结果更美观, 利用指标 $1, 2, 3$ 的轮换对称性, 并用 $r^i = \frac{1}{2} \epsilon^{i j k} J^{j k}, b^i = J^{0 i}$, 有,
		\begin{equation}
			\begin{dcases}
				\alpha = i (r^3 + i b^3) & A_\alpha = \frac{i}{2} ((r^1 + i b^1) + i (r^2 + i b^2)) \quad B_\alpha = \frac{i}{2} ((r^1 + i b^1) - i ( r^2 + i b^2)) \\
				\beta = i (r^3 - i b^3) & A_\beta = \frac{i}{2} ((r^1 - i b^1) + i (r^2 - i b^2)) \quad B_\beta = \frac{i}{2} ((r^1 - i b^1) - i (r^2 - i b^2))
			\end{dcases}
		\end{equation}
		
		\begin{tcolorbox}[title=calculation:]
			对于 $r^i$ 和 $b^i$, 有,
			\begin{equation}
				[r^i, r^j] = \epsilon^{i j k} r^k \quad [b^i, b^j] = - \epsilon^{i j k} r^k \quad [r^i, b^j] = \epsilon^{i j k} b^k
			\end{equation}
			因此, 令,
			\begin{equation}
				J^{(\pm)}_i = \frac{i}{2} (r^i \pm i b^i) \Longrightarrow \begin{dcases}
					[J^{(\pm)}_i, J^{(\pm)}_j] = i \epsilon_{i j k} J^{(\pm)}_k \\
					[J^{(+)}_i, J^{(-)}_j] = 0
				\end{dcases}
			\end{equation}
			有 $\mathrm{span}(J^{(\pm)}_i) = \mathfrak{su}(2)_\mathbb{C}$.
		\end{tcolorbox}
	\end{itemize}
\end{itemize}

\section{the \texorpdfstring{$(j_+, j_-)$}{(j+, j-)} representation of the Lorentz algebra}
\begin{itemize}
	\item $(j_+, j_-)$ representation, 也就是 $\text{spin-} j_+ \oplus \text{spin-} j_-$, 因此,
	\begin{equation}
		\begin{dcases}
			\pi_{(j_+, j_-)}(r^i) = - i (J^{(+)}_i \otimes I^{(-)} + I^{(+)} \otimes J^{(-)}_i) \\
			\pi_{(j_+, j_-)}(b^i) = - J^{(+)}_i \otimes I^{(-)} + I^{(+)} \otimes J^{(-)}_i
		\end{dcases}
	\end{equation}
	下文中, 按照惯例省略 $\pi_{(j_+, j_-)}$.
\end{itemize}

\subsection{the \texorpdfstring{$(\frac{1}{2}, 0)$}{(1/2, 0)} representation}
\begin{itemize}
	\item $(\frac{1}{2}, 0)$ rep. (also known as left-handed spinor) 中 (spin-$0$ 中 $J^{(0)}_i = 0$),
	\begin{equation}
		\begin{dcases}
			r^i = - i \Big( \frac{1}{2} \sigma_i \otimes 1 + I \otimes 0 \Big) = - \frac{i}{2} \sigma_i \\
			b^i = - \frac{1}{2} \sigma_i \otimes 1 + I \otimes 0 = - \frac{1}{2} \sigma_i
		\end{dcases}
	\end{equation}
	其中 $\sigma_i$ 的具体形式见 \eqref{10.1.14}.
	
	\item 对于 $P, T, P T$, 一个自然的定义是,
	\begin{equation}
		\Pi_{(\frac{1}{2}, 0)}(P) = \mathcolor{red}{(?)} \quad \Pi_{(\frac{1}{2}, 0)}(T) = \mathcolor{red}{(?)}
	\end{equation}
\end{itemize}

\subsection{the \texorpdfstring{$(\frac{1}{2}, 0) \oplus (0, \frac{1}{2})$}{(1/2, 0)+(0, 1/2)} representation}
\begin{itemize}
	\item $(\frac{1}{2}, 0) \oplus (0, \frac{1}{2})$ rep. (also known as Dirac spinor) 中,
	\begin{equation}
		\pi_{(\frac{1}{2}, 0) \oplus (0, \frac{1}{2})}(J^{\mu \nu}) = \begin{pmatrix}
			\pi_{(\frac{1}{2}, 0)}(J^{\mu \nu}) & 0 \\
			0 & \pi_{(0, \frac{1}{2})}(J^{\mu \nu})
		\end{pmatrix}
	\end{equation}
	展开写为,
	\begin{equation}
		r^i = - \frac{i}{2} \begin{pmatrix}
			\sigma_i & \\
			& \sigma_i
		\end{pmatrix} \quad b^i = \frac{1}{2} \begin{pmatrix}
			- \sigma_i & \\
			& \sigma_i
		\end{pmatrix}
	\end{equation}
	
	\item 对于 $P, T, P T$, 一个自然的定义是,
	\begin{equation}
		\Pi_{(\frac{1}{2}, 0) \oplus (0, \frac{1}{2})}(P) = \begin{pmatrix}
			& I \\
			I &
		\end{pmatrix} = I \otimes \tau_1 \quad \Pi_{(\frac{1}{2}, 0) \oplus (0, \frac{1}{2})}(T) = \mathcolor{red}{(?)}
	\end{equation}
\end{itemize}

\subsection{the \texorpdfstring{$(\frac{1}{2}, \frac{1}{2})$}{(1/2, 0)} representation}
\begin{itemize}
	\item $(\frac{1}{2}, \frac{1}{2})$ rep. (also known as vector) 中,
	\begin{equation}
		\begin{dcases}
			r^i = - i \Big( \frac{1}{2} \sigma_i \otimes I_{2 \times 2} + I_{2 \times 2} \otimes \frac{1}{2} \sigma_i \Big) \\
			b^i = - \frac{1}{2} \sigma_i \otimes I_{2 \times 2} + I_{2 \times 2} \otimes \frac{1}{2} \sigma_i
		\end{dcases}
	\end{equation}
	展开写为,
	\begin{align}
		& r^1 = - \frac{i}{2} \begin{pmatrix}
			\sigma_1 & I \\
			I & \sigma_1
		\end{pmatrix} \quad r^2 = - \frac{i}{2} \begin{pmatrix}
			\sigma_2 & - i I \\
			i I & \sigma_2
		\end{pmatrix} \quad r^3 = - \frac{i}{2} \begin{pmatrix}
			\sigma_3 + I & \\
			& \sigma_3 - I
		\end{pmatrix} \notag \\
		& b^1 = \frac{1}{2} \begin{pmatrix}
			- \sigma_1 & I \\
			I & - \sigma_1
		\end{pmatrix} \quad b^2 = \frac{1}{2} \begin{pmatrix}
			- \sigma_2 & - i I \\
			i I & - \sigma_2
		\end{pmatrix} \quad b^3 = \frac{1}{2} \begin{pmatrix}
			- \sigma_3 + I & \\
			& - \sigma_3 - I
		\end{pmatrix}
	\end{align}
\end{itemize}

\section{\texorpdfstring{$P, T$}{P, T} in representation}
\begin{itemize}
	\item 由于 $P, T, P T$ 与单位元 $I$ 不在同一个单连通分支
\end{itemize}
