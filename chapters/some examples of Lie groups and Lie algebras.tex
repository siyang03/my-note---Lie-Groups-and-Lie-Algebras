\chapter{some examples of Lie groups and Lie algebras}
\section{general linear groups and algebras}
\begin{itemize}
	\item $\mathrm{GL}(n, \mathbb{C}) = \{M \in \mathcal{M}_n(\mathbb{C}) | \det M \neq 0\}$.
	\begin{itemize}
		\item $\dim \mathrm{GL}(n, \mathbb{C}) = n^2$.
		
		\item $\mathrm{GL}(n, \mathbb{R})$ 有两个连通分支,
		\begin{equation}
			\mathrm{GL}(n, \mathbb{R}) = {\det}^{- 1}[(- \infty, 0)] \sqcup {\det}^{- 1}[(0, \infty)]
		\end{equation}
	\end{itemize}
	
	\item $\mathfrak{gl}(n, \mathbb{C}) = \mathcal{M}_n(\mathbb{C})$.
	
	\item the left-invariant vector field at $g$ is,
	\begin{equation}
		{(A_g)^i}_j = {x^i}_k(g) {(A_e)^k}_j
	\end{equation}
	and the Lie bracket is,
	\begin{equation}
		[A, B] = A B - B A
	\end{equation}
	
	\begin{tcolorbox}[title=proof:]
		for general linear group, $\tensor{x}{^i_j}(g h) = \tensor{x}{^i_k}(g) \tensor{x}{^k_j}(h)$.
		
		so, the pushforward of the left transformation is,
		\begin{equation}
			L_{g *} (A \Big|_e) \tensor{x}{^i_j} \Big|_g = A(\tensor{y}{^i_j}) \Big|_e
		\end{equation}
		where $\tensor{y}{^i_j}(h) = (L_g^* \tensor{x}{^i_j})(h) = \tensor{x}{^i_k}(g) \tensor{x}{^k_j}(h)$, so we have,
		\begin{equation}
			A(\tensor{y}{^i_j}) \Big|_e = A \Big|_e (\tensor{x}{^k_l}) \underbrace{\frac{\partial \tensor{y}{^i_j}}{\partial \tensor{x}{^k_l}} \Big|_e}_{= \tensor{x}{^i_m}(g) \delta^m_k \delta^l_j} = \tensor{x}{^i_k}(g) A \Big|_e (\tensor{x}{^k_j})
		\end{equation}
		
		\noindent\rule[0.5ex]{\linewidth}{0.5pt} % horizontal line
		
		\begin{align}
			\tensor{[A, B]}{^i_j} &= (d\tensor{x}{^i_j})_a (A^b \partial_b B^a - B^b \partial_b A^a) \notag \\
			&= \tensor{A}{^k_l} \frac{\partial}{\partial \tensor{x}{^k_l}} \tensor{B}{^i_j} - \tensor{B}{^k_l} \frac{\partial}{\partial \tensor{x}{^k_l}} \tensor{A}{^i_j}
		\end{align}
		注意 $\tensor{(A_g)}{^i_j} = \tensor{x}{^i_k}(g) \tensor{(A_e)}{^k_j}$, 所以,
		\begin{equation}
			\frac{\partial}{\partial \tensor{x}{^k_l}} (\tensor{A}{^i_j}) \Big|_g = \underbrace{\frac{\partial}{\partial \tensor{x}{^k_l}} (\tensor{x}{^i_m}(g))}_{= \delta^i_k \delta^l_m} \tensor{(A_e)}{^m_j} = \delta^i_k \tensor{(A_e)}{^l_j}
		\end{equation}
		代入得到,
		\begin{align}
			\tensor{[A, B]}{^i_j} \Big|_g &= \tensor{(A_g)}{^k_l} \delta^i_k \tensor{(B_e)}{^l_j} - \tensor{(B_g)}{^k_l} \delta^i_k \tensor{(A_e)}{^l_j} \notag \\
			&= \tensor{x}{^i_k}(g) (\tensor{A}{^k_l} \tensor{B}{^l_j} - \tensor{B}{^k_l} \tensor{A}{^l_j})
		\end{align}
	\end{tcolorbox}
\end{itemize}

\section{special linear groups and algebras}
\begin{itemize}
	\item $\mathrm{SL}(n, \mathbb{C}) = \{M \in \mathrm{GL}(n, \mathbb{C}) | \det M = 1\}$.
	
	\item $\mathfrak{sl}(n, \mathbb{C}) = \{A \in \mathcal{M}_n(\mathbb{C}) | \mathrm{tr} A = 0\}$.
\end{itemize}

\section{the Lorentz group and the Lorentz algebra}
\subsection{indefinite orthogonal groups}
\begin{itemize}
	\item $\mathrm{O}(p, q) = \{\Lambda \in \mathcal{M}_n(\mathbb{R}) | \Lambda^T \eta \Lambda = \eta\}$ is called the \href{https://en.wikipedia.org/wiki/Indefinite_orthogonal_group}{indefinite orthogonal group}, where $n = p + q$ and,
	\begin{equation}
		\eta = \mathrm{diag}(\underbrace{+ 1, \cdots, + 1}_{p}, \underbrace{- 1, \cdots, - 1}_{q})
	\end{equation}
	\begin{itemize}
		\item 将 $\lambda$ 视作一组列向量 $(\lambda_1, \cdots, \lambda_n)$, 那么,
		\begin{equation}
			\eta(\lambda_\mu, \lambda_\nu) = \eta_{\mu \nu}
		\end{equation}
		即 $n$ 个互相正交的向量.
		
		\item $\dim \mathrm{O}(p, q) = \frac{n (n - 1)}{2}$.
		
		\item 可以证明, 对于,
		\begin{equation}
			\Lambda = \begin{pmatrix}
				A & B \\
				C & D
			\end{pmatrix}
		\end{equation}
		有 $\det \Lambda = \frac{\det A}{\det D}$, 且 $|\det A|, |\det D| \geq 1$.
		
		\begin{tcolorbox}[title=proof:]
			分块矩阵满足,
			\begin{equation} \label{11.3.4}
				\begin{dcases}
					A^T B = C^T D \\
					A^T A - C^T C = I_{p \times p} \\
					D^T D - B^T B = I_{q \times q}
				\end{dcases}
			\end{equation}
			如果 $\det A \neq 0$, 那么,
			\begin{equation}
				\det \Lambda = \det(A) \det(D - C A^{- 1} B)
			\end{equation}
			对 \eqref{11.3.4} 的第一行做变换, 得到,
			\begin{equation}
				A^{- 1} = C^{- 1} (D^T)^{- 1} B^T \Longrightarrow C A^{- 1} B = (D^T)^{- 1} B^T B
			\end{equation}
			再代入 \eqref{11.3.4} 的第三行, 得到 $C A^{- 1} B = D - (D^T)^{- 1}$, 所以...
			
			\noindent\rule[0.5ex]{\linewidth}{0.5pt} % horizontal line
			
			由 \eqref{11.3.4} 的第二行,
			\begin{equation}
				{\det}^2 A = \det(I + C^T C) \overset{\textcolor{red}{(?)}}{\geq} 1
			\end{equation}
		\end{tcolorbox}
	\end{itemize}
	
	\item $\mathrm{O}(p, q)$ 具有如下子群,
	\begin{equation}
		\begin{dcases}
			\mathrm{SO}(p, q) = \{\Lambda \in \mathrm{O}(p, q) | \det \Lambda = 1\} \\
			\mathrm{SO}_+(p, q) = \{\Lambda \in \mathrm{SO}(p, q) | \det A \geq 1\} \\
			\mathrm{O}_+(p, q) = \{\Lambda \in \mathrm{O}(p, q) | \det A \geq 1\} \\
			\mathrm{O}_-(p, q) = \{\Lambda \in \mathrm{O}(p, q) | \det D \geq 1\}
		\end{dcases}
	\end{equation}
	且有如下四个连通分支,
	\begin{equation}
		\mathrm{SO}_\pm(p, q) \quad \text{and} \quad \mathrm{O'}_\pm(p, q) = \{\det \Lambda = - 1, \det A \geq 1 \ \text{or} \ \det A \leq - 1\}
	\end{equation}
\end{itemize}

\subsection{the Lorentz group}
\begin{itemize}
	\item $\mathrm{L} = \mathrm{O}(3, 1)$ is called the Lorentz group.
\end{itemize}

\subsection{the Lorentz algebra}
\begin{itemize}
	\item $\mathfrak{so}(3, 1) = \{A \in \mathcal{M}_4(\mathbb{R}) | A^T = - \eta A \eta\} \simeq \mathfrak{so}(4)$.
	
	\item 考虑 $\mathfrak{so}(4, \mathbb{C})$ 的 Dynkin diagram, $D_2$, (见 section \ref{6.7}), 可见 $\mathfrak{so}(4, \mathbb{C}) \simeq \mathfrak{sl}(2, \mathbb{C}) \oplus \mathfrak{sl}(2, \mathbb{C})$.
	\begin{itemize}
		\item 因此, $\mathfrak{so}(3, 1)$ 的 irreducible rep. 是 $\text{spin-} j_1 \oplus \text{spin-} j_2$, 用 $(j_1, j_2)$ 表示.
	\end{itemize}
\end{itemize}

\section{unitary groups and algebras}
\begin{itemize}
	\item $\mathrm{U}(n) = \{U \in \mathrm{GL}(n, \mathbb{C}) | U^\dag U = I\}$.
	\begin{itemize}
		\item $\dim \mathrm{U}(n) = n^2$.
		
		\item $\mathrm{U}(n)$ is connected.
	\end{itemize}
	
	\item $\mathfrak{u}(n) = \{A \in \mathcal{M}_n(\mathbb{C}) | A^\dag = - A\}$.
\end{itemize}

\section{special unitary groups and algebras}
\begin{itemize}
	\item $\mathrm{SU}(n) = \{U \in \mathrm{GL}(n, \mathbb{C}) | U^\dag U = I, \det U = 1\}$.
	\begin{itemize}
		\item $\dim \mathrm{SU}(n) = n^2 - 1$.
	\end{itemize}
	
	\item $\mathfrak{su}(n) = \{A \in \mathcal{M}_n(\mathbb{C}) | A^\dag = - A, \mathrm{tr} A = 0\}$.
\end{itemize}

\section{symplectic groups}
\begin{itemize}
	\item $\mathrm{Sp}(2 n, \mathbb{C}) = \{A \in \mathcal{M}_{2 n}(\mathbb{C}) | - \Omega A^T \Omega = A^{- 1}\}$, where,
	\begin{equation}
		\Omega = \begin{pmatrix}
			0 & I \\
			- I & 0
		\end{pmatrix}
	\end{equation}
	\begin{itemize}
		\item $\dim \mathrm{Sp}(2 n, \mathbb{C}) = 2 n (2 n + 1)$.
	\end{itemize}
\end{itemize}

\section{the representations of \texorpdfstring{$\mathfrak{sl}(3, \mathbb{C})$}{sl(3, C)}}
\begin{itemize}
	\item this section, we are going to discuss the classification of the irreducible rep. of $\mathrm{SU}(3)$ and $\mathfrak{sl}(2, \mathbb{C})$.
	
	\item $\mathfrak{sl}(3, \mathbb{C}) \simeq \mathfrak{su}(3)_\mathbb{C}$.
	
	\item $\mathrm{SU}(m)$ are \textbf{simply connected}, \textbf{compact} Lie groups.
	\begin{itemize}
		\item according to section \ref{5.1.1}, 单连通李群 (的表示) 完全由其李代数 (的表示) 决定.
		
		rep. of $\mathfrak{sl}(3, \mathbb{C}) \overset{\text{restrict to}}{\Longrightarrow}$ rep. of $\mathfrak{su}(3) \overset{\text{simple connectedness}}{\Longrightarrow}$ rep. of $\mathrm{SU}(3)$.
		
		\item according to section \ref{5.2}, $\Pi$ is irreducible $\iff \pi$ is irreducible.
		
		and $\mathrm{SU}(3)$ is \textbf{compact}, so it has complete reducibility property $\Longrightarrow$ rep. of $\mathfrak{sl}(3, \mathbb{C})$ is \textbf{completely reducible}. 可见, 半单李代数的表示都是 completely reducible.
	\end{itemize}
\end{itemize}
